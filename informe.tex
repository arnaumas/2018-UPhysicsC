\documentclass[12pt, a4paper, twocolumn, twoside]{article}

\usepackage[utf8]{inputenc}
\usepackage[T1]{fontenc}
\usepackage[english]{babel}
\usepackage{lmodern}
\usepackage{geometry}
\usepackage{hyperref}
\usepackage[dvipsnames]{xcolor}
\usepackage[bf,sf,small,pagestyles]{titlesec}
\usepackage{titling}
\usepackage{abstract}
\usepackage[font={footnotesize, sf}, labelfont=bf]{caption} 
\usepackage{siunitx}
\usepackage{graphicx}
\usepackage{booktabs}
\usepackage{amsmath,amssymb}
\usepackage{physics}
\usepackage[sort]{cleveref}
\usepackage{enumitem}
\usepackage{bibtopic}
\usepackage{chemformula}
\usepackage{mhchem}

\geometry{
	a4paper,
	right = 2.5cm,
	left = 2.5cm,
	bottom = 3cm,
	top = 3cm
}

\hypersetup{
	colorlinks,
	linkcolor = {red!50!blue},
	citecolor = {red!50!blue},
	linktoc = page
}

\numberwithin{table}{section}
\numberwithin{figure}{section}
\numberwithin{equation}{section}

\graphicspath{{./figs/}}

% Unitats
\sisetup{
	inter-unit-product = \ensuremath{ \, },
	allow-number-unit-breaks = true,
	detect-family = true,
	list-units = single,
	math-celsius = {}^{\circ}\kern-\scriptspace C
}

\DeclareMathAlphabet{\mathsfit}{T1}{\sfdefault}{\mddefault}{\sldefault}

\newcommand{\Z}{\mathbb{Z}}
\renewcommand{\vec}[1]{\mathbf{#1}}
\newcommand{\N}{\mathbb{N}}
\newcommand{\R}{\mathbb{R}}
\newcommand{\Ry}{\mathit{Ry}}
\newcommand{\inn}[2]{\left\langle #1 , #2 \right\rangle}
\newcommand{\proj}[1]{\ketbra{#1}{#1}}
\newcommand{\parbreak}{
	\begin{center}
		--- $\ast$ ---
	\end{center} 
}
\makeatletter
\newcommand*{\defeq}{\mathrel{\rlap{%
    \raisebox{0.3ex}{$\m@th\cdot$}}%
  \raisebox{-0.3ex}{$\m@th\cdot$}}%
	=
}
\makeatother

\newpagestyle{pagina}{
	\headrule
	\sethead*{\bfseries \sffamily Compost Pile Sizes}{}{\theauthor}
	\footrule
	\setfoot*{}{}{\sffamily \thepage}
}
\renewpagestyle{plain}{
	\footrule
	\setfoot*{}{}{\sffamily \thepage}
}
\pagestyle{pagina}

\title{\sffamily \bfseries Problem B: Compost Pile Sizes}
\author{\sffamily Team 242}
\date{}

\begin{document}
\renewcommand{\abstractname}{}
\renewcommand{\absnamepos}{empty}
\begin{titlingpage}
 	\maketitle

\noindent \hrulefill \\
\begin{abstract}
\sffamily
The aim of this report is to analyze and study the process of composting at a small scale. We model a composting pile focusing on the balance between the heat output by the process of composting and the losses due to convection. To quantify the heat produced by the microbial population we use a logistic model for its evolution, with a temperature dependent specific growth rate. Additionally we consider two different kinds of microorganisms, mesophiles and thermophiles. 

The differential equations we obtain are not analaytically solvable, so we use numerical methods to find solutions for a particular set of parameters based on the current literature. We analyze the solutions obtained for the case of a cubic pile and a semispherical pile.    

Finally, we discuss the range of validity of the model and how it could be improved with further research.
\end{abstract}
\hrulefill
\end{titlingpage}

\begin{titlingpage}
	\sffamily
	\tableofcontents
\end{titlingpage}

\section{Introduction}
The average American produces \SI{10}{kg} of organic waste every month. This number grows bigger if we consider families, schools and restaurants. The rising global waste is an issue that upcoming generations, if not ours, will eventually have to face. Composting is an easy, clean and free way of transforming the organic waste we produce into rich and fertile soil, which can be a helpful step to a self-suficient lifestyle. It is then useful to gain as much insight into this process as possible, to optimize it and make it more attractive for the general population. 

The goal of this paper is to model and study the process of composting at a small scale, paying special attention to the biological processes that take place and how they impact the physical aspects of composting.We will model how the different microbial populations grow in the pile and how we can maximize their efficiency by simply changing the shape of the compost pile. 
\section{The composting process} 
Composting is defined as the biological oxidative decomposition of organic constituents of waste under controlled conditions. This process is mainly used to recycle organic matter into useful soil conditioner which can be used in gardening, agriculture and horticulture.

\subsection{Biological aspects}
Many species take part in the process of composting organic matter. A high proportion of them are, according to \cite{cornell}, bacteria (80 to 90\%) but we can also find actinomycetes, fungi, protozoa and rotifers. We distinguish between \emph{mesophilic} and \emph{thermophilic} microbes. The first thrive in moderate temperatures, between \SI{20}{\celsius} and \SI{40}{\celsius}, while the latter do so in a higher range. In general, the term thermophilic applies to microorganisms that live in temperatures up to \SI{120}{\celsius}, but in the case of composting only \emph{simply thermophiles}, which sustain temperatures of up to \SI{60}{\celsius}.

This distinction gives rise to three differentiated phases of the composting process:

\emph{Mesophilic phase}, in which mesophilic microbes proliferate rapidly in the organic matter and decompose readily degradable components such as sugars. This process releases a lot of heat and lasts for between one and five days, when the temperature reaches \SI{40}{\celsius} and this type of microbes become increasingly inhibited.

After the increase in temperature activates the thermophilic microbes, composting enters the \emph{thermophilic phase}, in which more complex substances such as cellulose, protein and fats are decomposed at a high rate. This biooxidative activity produces \ch{CO2} and \ch{H2O} and releases heat. The high temperatures (\SIrange{40}{60}{\celsius}) destroy undesirable elements such as pathogens, fly larvae and weed seeds, thus sanitizing the compost. This phase lasts for several weeks or months, depending on the size of the system and the composition of the mixture.

Finally there is the \emph{cooling and maturation phase}, during which, as microbial populations stabilize, decomposition rate decreases and temperature declines, allowing mesophilic microorganisms to recolonize the compost. In the final process of stabilization and mineralization, the most stable organic compounds are decomposed and the compost becomes a fertile final product at an ambient temperature.
 
\subsection{Physical aspects}
\subsubsection{Temperature profile}\label{sec:temperatura}
The nature of the composting process leads to a characteristic temperature profile. A typical experimental temperature curve, such as the one in \cref{fig:corba model}, has a steep slope at the beginning, corresponding to the rapid output of heat during the growth of the first mesophilic populations. In some cases, there can be a characteristic dip in temperature at around \SI{40}{\celsius}. This is because at this point the mesophilic populations is reaching stagnation, and if the growth rate of the thermophilic population is still small, the combination of these two phenomena leads to a period of low microbial activity, and therefore a low power output. We can then identify the transition between the mesophilc and thermophilic stages. Finally, once thermophilic activity ceases, convective heat loss becomes dominant and the temperature lowers steadily, signaling the cooling and maturation and phase. 

\begin{figure}[htb]
	\sffamily \footnotesize \centering
	\includegraphics[scale=0.55]{model-T.jpg}
	\caption{Temperature curve for a composting process, \cite{dummies}}
	\label{fig:corba model}
\end{figure}

\subsubsection{Other aspects}
\begin{itemize}
	\item \textbf{Moisture.} Decomposition occurs more rapidly in the thin water films in the surface of the organic particles. Optimal moisture is found to be between 50 and 60\%.
	\item  \textbf{Aeration.} Oxygen is essential for the metabolism and respiration of aerobic organisms and for oxidizing the organic molecules of the waste material. During the process, \ch{O2} is consumed while \ch{CO2} is released. Various methods can be used to maintain the aerobic conditions necessary for the composting process to occur: aeration pipes, air holes, forced air flow or mechanical turning of the pile.
	\item  \textbf{Size of the compost system.} The size of the pile must be small enough to allow aeration through the pile, yet it needs to be of a sufficient size to prevent dissipation of heat.
	\end{itemize}

\section{Development of the model}
\subsection{General approach}
We will base our model on the energy balance of the heat pile. We will consider the pile as an homogeneous system, therefore there will be no spatial dependence for the temperature nor will we consider diffusive processes of the oxygen and other fluids inside the pile. There are two main terms at play: the heat produced by the biochemical process of composting, \( \dot{Q}_c \) and the heat losses of the pile, \( \dot{Q}_l \). We can thus write
\begin{equation} \label{eq:first balance}
	\dot{Q} = \dot{Q}_c + \dot{Q}_l. 
\end{equation}

Given that we are assuming that the pile is homogeneous, we can write the left hand side of \cref{eq:first balance} as \( \dot{Q} = mc\dot{T} \), where \( m \) is the mass of the pile and \( c \) its specific heat capacity. 

As for the losses, given the temperature range we are considering, \SIrange{20}{70}{\celsius}, it seems reasonable to assume that radiation plays no role and that these are mainly due to convection. Since part of the pile will be in contact with the ground, there is conduction taking place. However we will disregard it given that it is a much slower process than convection.  Thus we may write \( \dot{Q}_l = -hA\left(T - T_\text{env}\right) \), where \( h \) is the heat transfer coefficient of air, in \si{W.m^{-1}.K^{-1}}, \( A \) is the exposed surface area of the pile and \( T_\text{env} \) is the outside temperature. We will introduce a sinusoidal time dependence in the ambient temperature to model the given day-night shift between \SI{5}{\celsius} and \SI{20}{\celsius} so 
\begin{align} \label{eq:temp ambient}
	\begin{aligned}
		T_\text{env}(t)  & = \frac{T_\text{max} + T_\text{min}}{2} \\
									 	 & + \frac{T_\text{max} - T_\text{min}}{2} \sin{\left(\frac{2\pi t}{24}\right)},
	\end{aligned}
\end{align}
where \( t \) is the time in hours. 

If the chemical processes involved in composting have an exothermicity of \( Q_b \) then we can write	\( \dot{Q}_c = mQ_b \dot{B} \), where \( B \) is the percentage of mass of the microbial population. The task then is to determine an expression for \( \dot{B} \) that accurately describes the kinetic properties of microbial growth in composting processes.

\subsection{Determining the microbial growth rate}
It seems reasonable to assume that the evolution of the population should be logistic, given that composting entails the depletion of the organic compounds that serve as sustenance to the population. However, the growth should also be tightly coupled to the temperature, so that it reflects the temperature range in which the various populations thrive. The process that limits the growth can be though of as an enzymatic reaction between the two states of the microbes: active, in which they are consuming organic matter and so reproducing, and inactive, in which they are not, as described in \cite{85bio}. \cite{biokinetics} details the kinetics of this kind of processes. Thus the effect of temperature on microbial activity can be evaluated by considering the activation-inactivation reaction,
\begin{equation*}
    \ce{A <=>[\ce{k1}][\ce{k2}] I}. 
\end{equation*}
We have that $\dot{B}=k_1[A]$, since \( [A] \) is the fraction of the microbial population that is reproducing, and $\dot{B}= -k_2[I]$. Additionally, \( [A] + [I] = B \).  By combining these last three equations we have that
\begin{equation} \label{eq:cinetica 1}
	\left(\frac{1}{k_1}-\frac{1}{k_2}\right)\dot{B}=B.
\end{equation}
and 
\begin{equation} \label{eq:cinetica 2}
	k_1[A]+k_2[I]=0.
\end{equation}

If the forward and inverse reactions follow Arrhenius kinetics, we can write $k_i=P_ie^{-E_i/RT}$, where $P_i$ are the empirical pre-exponential factors, $E_i$ the respective activation energies, $R$ the universal gas constant and $T$ the temperature. This, combined with \cref{eq:cinetica 1,eq:cinetica 2} gives the temperature dependance of the specific growth rate \( \mu(T) \), 
\begin{equation} \label{eq:mumax}
    \mu(T)=\frac{A_1e^{-E_1/RT}}{1+A_2e^{-E_2/RT}}
\end{equation}
using the approximation $k_2\gg k_1$. \( A_1 \) is \( -P_1 \) and \( A_2 \) is \( P_2/P_1 \).

We now have an expression governing the evolution of the bacterial population,
\begin{equation}\label{eq:edo B}
	\dot{B} = \mu(T)B\left(1 - \frac{B}{B_\text{max}}\right),
\end{equation}
which gives us a specific form of the heat generated by the biological process. This form for the biologically generated heat appears in \cite{semenov}. There are additional terms that could be considered. \cite{semenov} uses the heat generated through the microbes' consumption of cellulose. We found that it is, however, much smaller when compared to the term we have obtained.  

Hence, we can write the differential equation that governs \( T \): 
\begin{equation} \label{eq:edo T}
	\begin{aligned}
		mc\dot{T} & = -hA\left(T - T_\text{env}(t)\right) \\
							& + Q_bm\mu(T)B(t)\left(1 - \frac{1}{B_\text{max}}\right). 
	\end{aligned}
\end{equation}
	This is coupled to \cref{eq:edo B}, which governs the growth of the mictrobial population.  

	This equation, however, only accounts for one microbial population. We have mentioned before that composting is a process that takes place in two different stages because there are two different kinds of organisms at play. A way of accounting for this fact in our model is to decouple \( B \) into \( B_1 \), the proportion of mesophilic microorganisms, and \( B_2 \), the proportion of thermophilic microorganisms, each with the correspoding exothermicities and specific growth rates. 

\subsection{Choice of parameters}
\cite{saucedo} gives the preexponential factors, activation energies for the fungus \textit{Aspergillus niger}. This species is mesophilic and proliferates in vegetable waste, specially lettuce, tomato and lemon. This means that we can certainly use this data since we are modeling the composting of domestic vegetable waste. 

No equivalent set of data for thermophilic microbes was readily available. For our model we extrapolated the corresponding parameters for a thermophilic population. Given that the thermophilic phase usually is longer and entails a smaller temperature change than the mesophilic phase, the specific growth rate and exothermicity of thermophiles should be smaller than those of mesophiles. 

In \cite{saucedo}, there is no distinction between mesophiles and thermophiles. It provides values for the initial and maximum concentration for the total microbial population. It seems reasonable to split these values between the mesophilic and thermophilic populations, since we could not find specific data for the ratio of mesophiles to thermophiles. Additionally, these values are given as percentages of the dry matter. Although it is widely known that the moisture content of the composting materials diminishes due to evaporation, we can argue, following \cite{niceassumptions}, that in our case it will remain constant since our pile is not physically isolated from the environment. We assume that the moisture content of vegetable waste is around 50\%. The density of the waste was assumed to be of \SI{500}{kg.m^{-3}}. 

All of these values are presented in \cref{tab:parametres}.

\section{Results and analysis}
The integration of the system of equations that govern \( T \), \( B_1 \) and \( B_2 \) was performed numerically in SageMath. 

\subsection{Analysis of a generic composting process}

\begin{figure}[htb]
	\sffamily \footnotesize \centering
	\input{figs/2-temperatura}
	\caption{Temperature evolution of the compost pile modeled with mesophilic and thermophilic populations}
	\label{fig:temp-mesos-termos}
\end{figure}

Considering a compost pile of a size legth of \SI{1}{m}, given the density we have considered, we obtain \SI{500}{kg} inside, we obtain the temperature profile in \cref{fig:temp-mesos-termos}. One can clearly observe the three different phases of the composting process. During the first 48 hours the temperature rises rapidly due to the burst in activity of the mesophilic population. This is also clear in \cref{fig:potencies}. We can see that the mesophilic population outputs the highest power during the first 24 hours but quickly decays as the mesophilic population reaches its saturation value and becomes stagnant.

After this, the thermophilic activity rises. Given that we modeled the thermophilic process to be much slower and less exothermic than the mesophilic one, the maximum output power is not nearly as high, nor is the decay as stark. This explains the slight dip in the temperature we observe after the mesophilic activity ceases: the power output by the thermophilic process in its initial stage is too low to counter the heat lost due to convection, therefore the temperature dips. After this, however, the temperature rises steadily to just under \SI{50}{\celsius}. We see that thermophilic activity diminishes after 12 days, when the population reaches its saturation value (\cref{fig:poblacions}). All of this is in agreement with the general characteristics of the composting process discussed in \cref{sec:temperatura}.

\begin{figure}[htb]
	\sffamily \footnotesize \centering
	% GNUPLOT: LaTeX picture with Postscript
\begingroup
\newcommand{\etiqueta}[1]{\setlength{\fboxsep}{0.75pt}\colorbox{white}{#1}}
  \makeatletter
  \providecommand\color[2][]{%
    \GenericError{(gnuplot) \space\space\space\@spaces}{%
      Package color not loaded in conjunction with
      terminal option `colourtext'%
    }{See the gnuplot documentation for explanation.%
    }{Either use 'blacktext' in gnuplot or load the package
      color.sty in LaTeX.}%
    \renewcommand\color[2][]{}%
  }%
  \providecommand\includegraphics[2][]{%
    \GenericError{(gnuplot) \space\space\space\@spaces}{%
      Package graphicx or graphics not loaded%
    }{See the gnuplot documentation for explanation.%
    }{The gnuplot epslatex terminal needs graphicx.sty or graphics.sty.}%
    \renewcommand\includegraphics[2][]{}%
  }%
  \providecommand\rotatebox[2]{#2}%
  \@ifundefined{ifGPcolor}{%
    \newif\ifGPcolor
    \GPcolortrue
  }{}%
  \@ifundefined{ifGPblacktext}{%
    \newif\ifGPblacktext
    \GPblacktextfalse
  }{}%
  % define a \g@addto@macro without @ in the name:
  \let\gplgaddtomacro\g@addto@macro
  % define empty templates for all commands taking text:
  \gdef\gplbacktext{}%
  \gdef\gplfronttext{}%
  \makeatother
  \ifGPblacktext
    % no textcolor at all
    \def\colorrgb#1{}%
    \def\colorgray#1{}%
  \else
    % gray or color?
    \ifGPcolor
      \def\colorrgb#1{\color[rgb]{#1}}%
      \def\colorgray#1{\color[gray]{#1}}%
      \expandafter\def\csname LTw\endcsname{\color{white}}%
      \expandafter\def\csname LTb\endcsname{\color{black}}%
      \expandafter\def\csname LTa\endcsname{\color{black}}%
      \expandafter\def\csname LT0\endcsname{\color[rgb]{1,0,0}}%
      \expandafter\def\csname LT1\endcsname{\color[rgb]{0,1,0}}%
      \expandafter\def\csname LT2\endcsname{\color[rgb]{0,0,1}}%
      \expandafter\def\csname LT3\endcsname{\color[rgb]{1,0,1}}%
      \expandafter\def\csname LT4\endcsname{\color[rgb]{0,1,1}}%
      \expandafter\def\csname LT5\endcsname{\color[rgb]{1,1,0}}%
      \expandafter\def\csname LT6\endcsname{\color[rgb]{0,0,0}}%
      \expandafter\def\csname LT7\endcsname{\color[rgb]{1,0.3,0}}%
      \expandafter\def\csname LT8\endcsname{\color[rgb]{0.5,0.5,0.5}}%
    \else
      % gray
      \def\colorrgb#1{\color{black}}%
      \def\colorgray#1{\color[gray]{#1}}%
      \expandafter\def\csname LTw\endcsname{\color{white}}%
      \expandafter\def\csname LTb\endcsname{\color{black}}%
      \expandafter\def\csname LTa\endcsname{\color{black}}%
      \expandafter\def\csname LT0\endcsname{\color{black}}%
      \expandafter\def\csname LT1\endcsname{\color{black}}%
      \expandafter\def\csname LT2\endcsname{\color{black}}%
      \expandafter\def\csname LT3\endcsname{\color{black}}%
      \expandafter\def\csname LT4\endcsname{\color{black}}%
      \expandafter\def\csname LT5\endcsname{\color{black}}%
      \expandafter\def\csname LT6\endcsname{\color{black}}%
      \expandafter\def\csname LT7\endcsname{\color{black}}%
      \expandafter\def\csname LT8\endcsname{\color{black}}%
    \fi
  \fi
    \setlength{\unitlength}{0.0500bp}%
    \ifx\gptboxheight\undefined%
      \newlength{\gptboxheight}%
      \newlength{\gptboxwidth}%
      \newsavebox{\gptboxtext}%
    \fi%
    \setlength{\fboxrule}{0.5pt}%
    \setlength{\fboxsep}{1pt}%
\begin{picture}(4420.00,3400.00)%
    \gplgaddtomacro\gplbacktext{%
      \csname LTb\endcsname%%
      \put(487,380){\makebox(0,0)[r]{\strut{}\num{0}}}%
      \csname LTb\endcsname%%
      \put(487,1070){\makebox(0,0)[r]{\strut{}\num{0.5}}}%
      \csname LTb\endcsname%%
      \put(487,1761){\makebox(0,0)[r]{\strut{}\num{1}}}%
      \csname LTb\endcsname%%
      \put(487,2451){\makebox(0,0)[r]{\strut{}\num{1.5}}}%
      \csname LTb\endcsname%%
      \put(487,3142){\makebox(0,0)[r]{\strut{}\num{2}}}%
      \csname LTb\endcsname%%
      \put(554,261){\makebox(0,0){\strut{}\num{0}}}%
      \csname LTb\endcsname%%
      \put(1043,261){\makebox(0,0){\strut{}\num{2}}}%
      \csname LTb\endcsname%%
      \put(1531,261){\makebox(0,0){\strut{}\num{4}}}%
      \csname LTb\endcsname%%
      \put(2020,261){\makebox(0,0){\strut{}\num{6}}}%
      \csname LTb\endcsname%%
      \put(2508,261){\makebox(0,0){\strut{}\num{8}}}%
      \csname LTb\endcsname%%
      \put(2997,261){\makebox(0,0){\strut{}\num{10}}}%
      \csname LTb\endcsname%%
      \put(3485,261){\makebox(0,0){\strut{}\num{12}}}%
      \csname LTb\endcsname%%
      \put(3974,261){\makebox(0,0){\strut{}\num{14}}}%
    }%
    \gplgaddtomacro\gplfronttext{%
      \csname LTb\endcsname%%
      \put(100,1830){\rotatebox{-270}{\makebox(0,0){\strut{}$\mathsfit P \ (\si{MW})$}}}%
      \csname LTb\endcsname%%
      \put(2386,83){\makebox(0,0){\strut{}$\mathsfit t \ (\si{days})$}}%
      \colorrgb{0.58,0.00,0.83}%%
      \put(920,1485){\rotatebox{-87}{\makebox(0,0){\strut{}\etiqueta{\footnotesize Mesophiles}}}}%
      \colorrgb{1.00,0.65,0.00}%%
      \put(1775,615){\rotatebox{12}{\makebox(0,0){\strut{}\etiqueta{\footnotesize Thermophiles}}}}%
    }%
    \gplbacktext
    \put(0,0){\includegraphics{2-potencies}}%
    \gplfronttext
  \end{picture}%
\endgroup

	\caption{Power output of the mesophilic and thermophilic reactions}
	\label{fig:potencies}
\end{figure}

After this point, the pile is subject solely to convective losses, and its temperature slowly decreases to the ambient temperature. It must be noted that the temperature does suffer small fluctuations ---within \SI{1}{\celsius}--- due to the oscillation of the ambient temperature. However the oscillations of the outside temperature are too fast ---they have a period of 1 day--- to have a significant effect on the temperature of the pile.  

\begin{figure}[htb]
	\sffamily \footnotesize \centering
	% GNUPLOT: LaTeX picture with Postscript
\begingroup
\newcommand{\etiqueta}[1]{\setlength{\fboxsep}{0.75pt}\colorbox{white}{#1}}
  \makeatletter
  \providecommand\color[2][]{%
    \GenericError{(gnuplot) \space\space\space\@spaces}{%
      Package color not loaded in conjunction with
      terminal option `colourtext'%
    }{See the gnuplot documentation for explanation.%
    }{Either use 'blacktext' in gnuplot or load the package
      color.sty in LaTeX.}%
    \renewcommand\color[2][]{}%
  }%
  \providecommand\includegraphics[2][]{%
    \GenericError{(gnuplot) \space\space\space\@spaces}{%
      Package graphicx or graphics not loaded%
    }{See the gnuplot documentation for explanation.%
    }{The gnuplot epslatex terminal needs graphicx.sty or graphics.sty.}%
    \renewcommand\includegraphics[2][]{}%
  }%
  \providecommand\rotatebox[2]{#2}%
  \@ifundefined{ifGPcolor}{%
    \newif\ifGPcolor
    \GPcolortrue
  }{}%
  \@ifundefined{ifGPblacktext}{%
    \newif\ifGPblacktext
    \GPblacktextfalse
  }{}%
  % define a \g@addto@macro without @ in the name:
  \let\gplgaddtomacro\g@addto@macro
  % define empty templates for all commands taking text:
  \gdef\gplbacktext{}%
  \gdef\gplfronttext{}%
  \makeatother
  \ifGPblacktext
    % no textcolor at all
    \def\colorrgb#1{}%
    \def\colorgray#1{}%
  \else
    % gray or color?
    \ifGPcolor
      \def\colorrgb#1{\color[rgb]{#1}}%
      \def\colorgray#1{\color[gray]{#1}}%
      \expandafter\def\csname LTw\endcsname{\color{white}}%
      \expandafter\def\csname LTb\endcsname{\color{black}}%
      \expandafter\def\csname LTa\endcsname{\color{black}}%
      \expandafter\def\csname LT0\endcsname{\color[rgb]{1,0,0}}%
      \expandafter\def\csname LT1\endcsname{\color[rgb]{0,1,0}}%
      \expandafter\def\csname LT2\endcsname{\color[rgb]{0,0,1}}%
      \expandafter\def\csname LT3\endcsname{\color[rgb]{1,0,1}}%
      \expandafter\def\csname LT4\endcsname{\color[rgb]{0,1,1}}%
      \expandafter\def\csname LT5\endcsname{\color[rgb]{1,1,0}}%
      \expandafter\def\csname LT6\endcsname{\color[rgb]{0,0,0}}%
      \expandafter\def\csname LT7\endcsname{\color[rgb]{1,0.3,0}}%
      \expandafter\def\csname LT8\endcsname{\color[rgb]{0.5,0.5,0.5}}%
    \else
      % gray
      \def\colorrgb#1{\color{black}}%
      \def\colorgray#1{\color[gray]{#1}}%
      \expandafter\def\csname LTw\endcsname{\color{white}}%
      \expandafter\def\csname LTb\endcsname{\color{black}}%
      \expandafter\def\csname LTa\endcsname{\color{black}}%
      \expandafter\def\csname LT0\endcsname{\color{black}}%
      \expandafter\def\csname LT1\endcsname{\color{black}}%
      \expandafter\def\csname LT2\endcsname{\color{black}}%
      \expandafter\def\csname LT3\endcsname{\color{black}}%
      \expandafter\def\csname LT4\endcsname{\color{black}}%
      \expandafter\def\csname LT5\endcsname{\color{black}}%
      \expandafter\def\csname LT6\endcsname{\color{black}}%
      \expandafter\def\csname LT7\endcsname{\color{black}}%
      \expandafter\def\csname LT8\endcsname{\color{black}}%
    \fi
  \fi
    \setlength{\unitlength}{0.0500bp}%
    \ifx\gptboxheight\undefined%
      \newlength{\gptboxheight}%
      \newlength{\gptboxwidth}%
      \newsavebox{\gptboxtext}%
    \fi%
    \setlength{\fboxrule}{0.5pt}%
    \setlength{\fboxsep}{1pt}%
\begin{picture}(4420.00,3400.00)%
    \gplgaddtomacro\gplbacktext{%
      \csname LTb\endcsname%%
      \put(435,380){\makebox(0,0)[r]{\strut{}\num{0}}}%
      \csname LTb\endcsname%%
      \put(435,743){\makebox(0,0)[r]{\strut{}\num{0.01}}}%
      \csname LTb\endcsname%%
      \put(435,1105){\makebox(0,0)[r]{\strut{}\num{0.02}}}%
      \csname LTb\endcsname%%
      \put(435,1468){\makebox(0,0)[r]{\strut{}\num{0.03}}}%
      \csname LTb\endcsname%%
      \put(435,1830){\makebox(0,0)[r]{\strut{}\num{0.04}}}%
      \csname LTb\endcsname%%
      \put(435,2193){\makebox(0,0)[r]{\strut{}\num{0.05}}}%
      \csname LTb\endcsname%%
      \put(435,2555){\makebox(0,0)[r]{\strut{}\num{0.06}}}%
      \csname LTb\endcsname%%
      \put(435,2918){\makebox(0,0)[r]{\strut{}\num{0.07}}}%
      \csname LTb\endcsname%%
      \put(435,3280){\makebox(0,0)[r]{\strut{}\num{0.08}}}%
      \csname LTb\endcsname%%
      \put(502,261){\makebox(0,0){\strut{}\num{0}}}%
      \csname LTb\endcsname%%
      \put(997,261){\makebox(0,0){\strut{}\num{2}}}%
      \csname LTb\endcsname%%
      \put(1493,261){\makebox(0,0){\strut{}\num{4}}}%
      \csname LTb\endcsname%%
      \put(1988,261){\makebox(0,0){\strut{}\num{6}}}%
      \csname LTb\endcsname%%
      \put(2484,261){\makebox(0,0){\strut{}\num{8}}}%
      \csname LTb\endcsname%%
      \put(2979,261){\makebox(0,0){\strut{}\num{10}}}%
      \csname LTb\endcsname%%
      \put(3475,261){\makebox(0,0){\strut{}\num{12}}}%
      \csname LTb\endcsname%%
      \put(3970,261){\makebox(0,0){\strut{}\num{14}}}%
    }%
    \gplgaddtomacro\gplfronttext{%
      \csname LTb\endcsname%%
      \put(2360,83){\makebox(0,0){\strut{}$\mathsfit t \ (\si{days})$}}%
      \csname LTb\endcsname%%
      \put(997,1830){\rotatebox{85}{\makebox(0,0){\strut{}\etiqueta{\footnotesize Mesophiles}}}}%
      \csname LTb\endcsname%%
      \put(2484,1830){\rotatebox{65}{\makebox(0,0){\strut{}\etiqueta{\footnotesize Thermophiles}}}}%
    }%
    \gplbacktext
    \put(0,0){\includegraphics{2-bacteris}}%
    \gplfronttext
  \end{picture}%
\endgroup

	\caption{Evolution of the relative mass of the populations of mesophilic and thermophilic organisms}
	\label{fig:poblacions}
\end{figure}

\subsection{Analysis of the dimensions of the pile}
Given that we are considering composting on the domestic scale, the range of sizes of the pile is rather limited. If we assume a cubic pile, then a surface area of \SI{10}{m^2}	entails a side length of about \SI{1.3}{m}, which would be on the edge of practicality for the domestic scale. We will therefore only consider surface areas within this range. 

\begin{figure}[htb]
	\sffamily \footnotesize \centering
	\input{figs/2-temperatura-A}
	\caption{Temperature evolution of a cubic compost pile for various values of surface area}
	\label{fig:temp-arees}
\end{figure}

For a given value of surface area, we can compute the mass of the pile as \( m = \rho\left(\frac{1}{5}A\right)^{\frac{3}{2}} \), since only five out of the six faces would suffer heat loss due to convection, the sixth one being in contact with the ground. We see that the temperature evolution during the mesophilic stage is essentially independent of the area, but not during the thermophilic stage. The maximum temperature ranges between \SI{50}{\celsius} and \SI{57}{\celsius}. The smaller the surface area, the higher the maximum temperature reached, which is reasonable since the smaller surface areas are less subject to convective losses. In any case, every size is able to reach temperatures high enough for composting to occur.  

\begin{figure}[htb]
	\sffamily \footnotesize \centering
	\input{figs/3-temperatura-A}
	\caption{Temperature evolution of a semispherical compost pile for various values of surface area}
	\label{fig:temp-semiesferes}
\end{figure}

To obtain \cref{fig:temp-semiesferes} we performed the same calculations considering a semispherical pile instead. The relationship between the mass and the surface area now is \( m = \frac{4\pi\rho}{3}\left(\frac{A}{2\pi}\right)^{\frac{3}{2}} \). We notice that in this case during both stages of composting the size does not play a higher role and the range of maximum temperatures achieved is much smaller than before. Every size is able to surpass \SI{50}{\celsius}, which was not the case when considering cubic piles. We conclude that a spherical size is more adequate given that it is less affected by the outside temperature. 

\section{Conclusions}
We have modeled the evolution of the temperature of a compost pile of a small size. We have seen that for various pile shapes, composting does take place, albeit with a wide range of maximum temperatures, between \SI{50}{\celsius} and \SI{60}{\celsius}. However, with a spherical pile, the surface to volume ratio is such that the temperature during the heating stages is independent of the size of the pile. If the pile is too small, however, convection is not sufficient to maintain the temperature below the threshold of \SI{60}{\celsius}, above which thermophiles cannot decompose the organic matter and composting does not take place. And if it is too large, the convective losses are too large and the pile does not reach \SI{40}{\celsius}, and so the thermophilic stage does not start and composting does not take place either. For the cubic pile, this range of volumes is between \SI{0.72}{m^3} and \SI{4.78}{m^3}.   

\subsection{Strengths and weaknesses}
Our model is reasonably simple, given that it doesnot take into account the inside of the pile, and is only concerned with the global heat balance. In spite of this we have managed to obtain sensible temperature profiles for a compost pile on the domestic scale. It also clearly shows the dependence of the composting process on the volume to surface ratio.

Considering the different types of microbial populations is not the standard approach to modelling composting processes. This gives a profile that is similar to many of the experimentally obtained temperature profiles, which in turn allows us to extract information on the duration and temperature of every phase, which determines the quality of the resulting fertilizer. 

However, considering the pile to be homogenous does not allow us to consider relevant phenomena such as the evaporation of water and consequent loss of humidity and heat as well the diffusion processes of oxygen inside the pile, which impacts the activity of the microorganisms ---composting is a strongly aerobic process, a lack of oxygen can lead to fermentation, an anaerobic process, which is undesirable---.  

Our model is tuned to the scale we are considering. If we apply it to larger scales it is not as useful, since we do not consider interior processes that become relevant as the pile grows bigger. 

\subsection{Further work}
During our research we encountered a plethora of different approaches to the modeling of composting processes, see \cite{mason}, which, on the other side are heavily dependent on parameteres that need to be experimentally determined. 

In particular, there is no experimental data that focuses on the characteristics of a single microorganism population. It would be useful to have more insight on the differences between the behavior of mesophiles and thermophiles. 

Our modelling of ambient temperature, was perhaps oversimplified, in the sense that such wide oscillations with such a short period of time does not have a relevant impact on the temperature we obtain. A future improvement could be to model the daily fluctuations in temperature and humidity stochastically. 

\clearpage
\appendix
\onecolumn

\section{Model parameters}
\begin{table}[htb]
\sffamily \footnotesize \centering
\caption{Chosen parameters}
\label{tab:parametres}
\begin{tabular}{cSS}
		\toprule 
		{Parameter} & {Value (Mesophiles)} & {Value (Thermophiles)} \\
		\midrule 
		$Q_b$ & \SI{1}{MJ.kg^{-1}} & \SI{0.36}{MJ.kg^{-1}} \\
		$E_1$ & \SI{70.225}{kJ.mol^{-1}} & \SI{74.438}{kJ.mol^{-1}} \\
		$E_2$ & \SI{283.356}{kJ.mol^{-1}} & \SI{300.357}{kJ.mol^{-1}} \\
		$A_1$ & \SI{2.69e11}{h^{-1}} & \SI{1.34}{h^{-1}} \\
		$A_2$ & 1.3e47 & 3.9e47 \\
		$B_0$ & 0.25\% & 0.25\% \\ 
		$B_\text{max}$ & 7.5\% & 7.5\% \\
		\bottomrule
	\end{tabular}
\end{table}

\section{References and further reading}
\begin{btSect}{fonts}
	\bibliographystyle{ieeetr}
	\subsection*{References}	
	\btPrintCited
\end{btSect}

\begin{btSect}{fonts}
	\bibliographystyle{ieeetr}
	\subsection*{Further Reading}	
	\btPrintNotCited
\end{btSect}

\end{document}
