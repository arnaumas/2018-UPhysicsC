\documentclass[12pt, a4paper, twocolumn]{article}

\usepackage[utf8]{inputenc}
\usepackage[T1]{fontenc}
\usepackage[english]{babel}
\usepackage{lmodern}
\usepackage{geometry}
\usepackage{hyperref}
\usepackage[dvipsnames]{xcolor}
\usepackage[bf,sf,small,pagestyles]{titlesec}
\usepackage{titling}
\usepackage{abstract}
\usepackage[font={footnotesize, sf}, labelfont=bf]{caption} 
\usepackage{siunitx}
\usepackage{graphicx}
\usepackage{booktabs}
\usepackage{amsmath,amssymb}
\usepackage{physics}
\usepackage[sort]{cleveref}
\usepackage{enumitem}
\usepackage{bibtopic}
\usepackage{chemformula}
\usepackage{mhchem}

\geometry{
	a4paper,
	right = 2.5cm,
	left = 2.5cm,
	bottom = 3cm,
	top = 3cm
}

\hypersetup{
	colorlinks,
	linkcolor = {red!50!blue},
	citecolor = {red!50!blue},
	linktoc = page
}

\numberwithin{table}{section}
\numberwithin{figure}{section}
\numberwithin{equation}{section}

\graphicspath{{./figs/}}

% Unitats
\sisetup{
	inter-unit-product = \ensuremath{ \, },
	allow-number-unit-breaks = true,
	detect-family = true,
	list-units = single,
	math-celsius = {}^{\circ}\kern-\scriptspace C
}

\DeclareMathAlphabet{\mathsfit}{T1}{\sfdefault}{\mddefault}{\sldefault}

\newcommand{\Z}{\mathbb{Z}}
\renewcommand{\vec}[1]{\mathbf{#1}}
\newcommand{\N}{\mathbb{N}}
\newcommand{\R}{\mathbb{R}}
\newcommand{\Ry}{\mathit{Ry}}
\newcommand{\inn}[2]{\left\langle #1 , #2 \right\rangle}
\newcommand{\proj}[1]{\ketbra{#1}{#1}}
\newcommand{\parbreak}{
	\begin{center}
		--- $\ast$ ---
	\end{center} 
}
\makeatletter
\newcommand*{\defeq}{\mathrel{\rlap{%
    \raisebox{0.3ex}{$\m@th\cdot$}}%
  \raisebox{-0.3ex}{$\m@th\cdot$}}%
=
}
\makeatother

\newpagestyle{pagina}{
	\headrule
	\sethead*{\bfseries \sffamily Compost Pile Sizes}{}{\theauthor}
	\footrule
	\setfoot*{}{}{\sffamily \thepage}
}
\renewpagestyle{plain}{
	\footrule
	\setfoot*{}{}{\sffamily \thepage}
}
\pagestyle{pagina}

\title{\sffamily \bfseries Problem B: Compost Pile Sizes}
\author{\sffamily Team 242}
\date{}

\begin{document}
\renewcommand{\abstractname}{}
\renewcommand{\absnamepos}{empty}
\begin{titlingpage}
 \maketitle

\noindent \hrulefill \\
\begin{abstract}

\end{abstract}
\hrulefill
\end{titlingpage}

\begin{titlingpage}
\tableofcontents
\end{titlingpage}

\section{Introduction}
\cite{mason}

\section{The composting process}
Composting is defined as the biological oxidative decomposition of organic constituents in wastes (generally organic) under controlled conditions. This process is mainly used to recycle organic matter into useful soil conditioner which can be used in gardening, agriculture and horticulture.

\subsection{Microbes}
Many species take part in the process of composting organic matter. A high proportion of them are bacteria (80 to 90\%) but we can also find actinomycetes, fungi, protozoa and rotifers. The most important distinction between them in relation to their role in the composting process is the following:

\textbf{Mesophilic microbes:} Microorganisms that grow best in moderate temperatures, between \SI{20}{\celsius} and \SI{40}{\celsius}.

\textbf{Thermophilic microbes (simple):} Microorganisms that grow best in high temperatures, between \SI{40}{\celsius} and \SI{60}{\celsius}.

\subsection{Phases of the process}
During the process of composting, the substrate evolves through three principal phases:

\textbf{Mesophilic phase:} In the first stage, mesophilic microbes proliferate rapidly in the organic matter and decompose readily degradable components, such as sugars. This process releases a lot of heat and lasts for between one and five days, when the temperature raises up to \SI{40}{\celsius} and this type of microbes become increasingly inhibited.

\textbf{Thermophilic phase:} The raise in temperature activates the thermophilic microbes, which decompose cellulose, protein and fats at a high rate. This biooxidative activity produces \ch{CO2} and \ch{H2O} and releases heat. The high temperatures (\SIrange{40}{60}{\celsius}) lead to the destruction of animal and plant pathogens, fly larvae and weed seeds, sanitizing the compost. This phase lasts for several weeks or months, depending on the size of the system and the composition of the mixture.

\textbf{Cooling and maturation phase:} As microbial populations stabilize, decomposition rate decreases and temperature declines, allowing mesophilic microorganisms to recolonize the compost. In the final process of stabilization and mineralization, the most stable organic compounds are decomposed and the compost becomes a "humified" final product at an ambient temperature.
 
\subsection{Physical factors}
\textbf{Temperature:} The composting process produces heat as a by-product of the microbial breakdown of organic material. On the other hand, part of the heat is lost through conduction, convection, and radiation. The amount of heat depends on the size of the pile, its moisture content and aeration, as well as the ambient temperatures and the isolation of the pile. A typical experimental temperature curve has a steep slope at the beginning, corresponding to the growth of the first mesophilic populations. Having reached \SI{40}{\celsius}, the curve can present a small decreasing due to the stagnation of the mesophilic population if the growth rate of the thermophilic population is small. We can then identify the thermophilic stage, where the temperatures remain practically constant at their maximum. Finally, we can see the decreasing of the temperature corresponding to the cooling and maturation and phase.

\begin{figure}[htb]
	\sffamily \footnotesize \centering
	\includegraphics[scale=0.55]{model-T.jpg}
	\caption{Model of the temperature curve for a composting process}
	\label{fig:temp-arees}
\end{figure}

\textbf{Moisture:} Decomposition occurs more rapidly in the thin water films in the surface of the organic particles. Optimal moisture is found to be between 50 and 60\%.

\textbf{Aeration:} Oxygen is essential for the metabolism and respiration of aerobic organisms and for oxidizing the organic molecules of the waste material. During the process, \ch{O2} is consumed while \ch{CO2} is released. Various methods can be used to maintain the aerobic conditions necessary for the composting process to occur: aeration pipes, air holes, forced air flow or mechanical turning of the pile.

\textbf{Size of the compost system:} The size of the pile must be small enough to allow aeration through the pile, yet it needs to be of a sufficient size to prevent dissipation of heat.
% typical experimental temperature curve
\section{The model}
\subsection{Assumptions and considerations}
\subsubsection{Environment}

The compost pile will be exposed to day-night temperature shift, since it will be left outdoors. Thus, we will consider a varying room temperature between \SI{5}{\celsius} and \SI{20}{\celsius} as a sinusoidal function for our pile. In addition, we will let the leftovers begin at a \SI{20}{\celsius} temperature.  

\subsubsection{Compost}

We are trying to study the decomposition of kitchen vegetable waste, and hence we will not produce a big amount of compost. For a medium-size family, \si{30}{kg} seems an adequate monthly output of organic waste. 

For the sake of simplicity, we can assume that our pile will be homogeneous: we will mix the pile initially in a way such that we are not able to distinguish the solid, liquid and gas phase. This way we can also assume that the thermophysical properties will remain constant, including the porosity. (It might be thought that as the decomposition goes on, the gas phase will grow bigger and therefore the porosity will be reduced, but we will consider an average value of it.) %SOBRA? 

With regard to moisture, it is widely known that its content of the composting materials diminishes due to evaporation, nonetheless if the evaporation is compensated by the water being fed with the gas phase; the moisture content could be kept nearly constant \cite{niceassumptions}. We may assume then a moisture level of a 50\%, for it is a reasonable percentage considering only vegetable kitchen waste and its high amount of water. 

Any periodic aeration of the pile will be avoided. The dimensions of the compost pile are not big enough to require an external aeration system. Due to the low thermal conductance of the container and the possibility of some slow air flow through the holes of it, the compost will receive the oxygen it needs in order to keep up with the decomposition but without having a big effect on the temperature. In case any of our designs reached high temperatures (above \SI{60}{\celsius} or so), it will be discarded. 

As we were saying, the container of the pile will be thought as a HDPE cubic box not completely isolated (i.e. some air flow is permitted). This way we will avoid the lack of oxygen without losing the isolation. The walls of it will be thin, this way we can neglect the heat losses due to conduction, and will also grant a weaker isolation that will lead to smoother temperatures. 

\subsection{Numerical model}
We will ground our model of the evolution of a compost pile on its heat balance equation. There are two main terms at play: the heat produced by the biochemical process of composting, \( \dot{Q}_c \) and the heat losses of the pile, \( \dot{Q}_l \). We can thus write
\begin{equation} \label{eq:first balance}
	\dot{Q} = \dot{Q}_c + \dot{Q}_l. 
\end{equation}
The temperature within the compost pile will be considered constant throughout, as well as its heat capacity, and so the first term of \cref{eq:first balance} becomes \( \dot{Q} = mc\dot{T} \), where \( m \) is the mass of the pile and \( c \) its specific heat capacity. As for the losses, given the temperature range we are considering, \SIrange{20}{70}{\celsius}, it seems reasonable to assume that these are mainly due to convection. Radiation in this range of temperatures is perfectly negligible. Thus we may write \( \dot{Q}_l = -UA\left(T - T_\text{env}\right) \), where \( U \) is the overall heat transfer coefficient, in \si{W.m^{-1}.K^{-1}}, \( A \) is the total surface area of the pile and \( T_\text{env} \) is the outside temperature. We may also discard losses due to conduction, since we consider the walls of the pile to be thin.

Regarding the biochemical process of composting, we will look for the maximum specific growth rate of microorganisms. One could think this process as the proposed by Monod independent of the temperature, but it has already been studied and found to be applicable within only a limited range. However, we can solve this using a generalized temperature effect model if the influence of the cultivation temperature on the activity of the enzyme involved in the growth limiting reaction is taken into account\cite{85bio}. We will assume that the enzyme can exist in two possible states, an active and an inactive form, in equilibrium with each other\cite{biokinetics}. Thus the effect of temperature on enzyme activity can be evaluated by considering the activation-inactivation reaction:
\begin{equation}
    \ce{A <=>[\ce{k1}][\ce{k2}] I}
\end{equation}

We also have that $\dot{B}=-[A]k_1$ and $\dot{B}=[I]k_2$, which obey that $[A]+[I]=B$, where $B$ is the general concentration of microbes, $[A]$ the concentration of the activation form and $[I]$ the concentration of the inactivation form. By combining the last three equations we have that $\dot{B}(\frac{1}{k_1}-\frac{1}{k_2})=B$ and that $k_1[A]+k_2[I]=0$. Therefore, if we also have a look at Arrhenius equations, $k_i=Pe^{-E_i/RT}$, where $P$ is an empirical factor, $E_i$ the activation energy of the activation or inactivation process respectively, $R$ the universal gas constant and $T$ the temperature, we obtain 
\begin{equation} \label{eq:mumax}
    \mu(T)=\frac{A_1e^{-E_1/RT}}{1+A_2e^{-E_2/RT}}
\end{equation}
where we used $\mu(T)=\dot{B}/B$ and the approximation $k_2\gg k_1$ (i.e. the inactive reaction is the faster one).

Following this reasoning we can now write the first term on the right hand side of \cref{eq:first balance} as $\dot{Q_c}=Q_bm\mu(T)B\left(1-\frac{B}{B_{max}}\right)$, where $Q_b$ is the exothermicity of the decomposition reaction, in \SI{}{J}, $m$ is the mass of the compost and $\mu(T)B\left(1-\frac{B}{B_{max}}\right)=\dot{B}$ is the derivative of the mass of microbes, because we assume the microbes follow a logistic model of growth \cite{saucedo}. 

It must be said that we are neglecting a term in the biological generation of energy. The organic waste has a big proportion of cellulose, and one could think that its decomposition may also affect the released energy. Nonetheless, if we took a look into the pre-exponential factor of its Arrhenius decomposition reaction it would be appreciated that it is negligible compared to the microbial reaction.

Hence, we can rewrite \cref{eq:first balance} with the results we have discussed, shown in \cref{eq:second balance}
\begin{equation}\label{eq:second balance}
    \dot{Q}=Q_bm\mu(T)B\left(1-\frac{B}{B_{max}}\right)-UA(T -T_{env})
\end{equation}

\subsection{Experimental data}

We took our experimental data from the decomposition of vegetable waste (lettuce, tomato and lemon) by the mesophile \textit{Aspergillus niger} \cite{saucedo}. We obtained the pre-exponential factors for both active and inactive reactions in equilibrium, as well as the activation energies of both. In addition, we followed the initial and final concentration of microbes in our model. 
The idea was to extrapolate this particular case of decomposition to any mesophilic decomposition, always following the theoretical behaviour of temperatures and times: activating at room temperature and stopping at \SI{40}{\celsius}, where their activity reaches a maximum approximately one or two days later. From this extrapolation, we also managed to obtain the pre-exponential factors and activation energies for a thermophilic decomposition, which would activate at \SI{40}{\celsius} and would reach their peak activity at \SI{60}{\celsius}, paying special attention to the period and length of their activity. This data will be presented and discussed in the following section. 

\begin{figure}[htb]
	\sffamily \footnotesize \centering
	%\input{figs/2-temperatura}
	\caption{Temperature evolution of the compost pile modeled with mesophilic and thermophilic populations}
	\label{fig:temp-mesos-termos}
\end{figure}

\begin{figure}[htb]
	\sffamily \footnotesize \centering
	%\input{figs/2-temperatura-A}
	\caption{Temperature evolution of the compost pile for various values of surface area}
	\label{fig:temp-arees}
\end{figure}

\begin{figure}[htb]
	\sffamily \footnotesize \centering
	%% GNUPLOT: LaTeX picture with Postscript
\begingroup
\newcommand{\etiqueta}[1]{\setlength{\fboxsep}{0.75pt}\colorbox{white}{#1}}
  \makeatletter
  \providecommand\color[2][]{%
    \GenericError{(gnuplot) \space\space\space\@spaces}{%
      Package color not loaded in conjunction with
      terminal option `colourtext'%
    }{See the gnuplot documentation for explanation.%
    }{Either use 'blacktext' in gnuplot or load the package
      color.sty in LaTeX.}%
    \renewcommand\color[2][]{}%
  }%
  \providecommand\includegraphics[2][]{%
    \GenericError{(gnuplot) \space\space\space\@spaces}{%
      Package graphicx or graphics not loaded%
    }{See the gnuplot documentation for explanation.%
    }{The gnuplot epslatex terminal needs graphicx.sty or graphics.sty.}%
    \renewcommand\includegraphics[2][]{}%
  }%
  \providecommand\rotatebox[2]{#2}%
  \@ifundefined{ifGPcolor}{%
    \newif\ifGPcolor
    \GPcolortrue
  }{}%
  \@ifundefined{ifGPblacktext}{%
    \newif\ifGPblacktext
    \GPblacktextfalse
  }{}%
  % define a \g@addto@macro without @ in the name:
  \let\gplgaddtomacro\g@addto@macro
  % define empty templates for all commands taking text:
  \gdef\gplbacktext{}%
  \gdef\gplfronttext{}%
  \makeatother
  \ifGPblacktext
    % no textcolor at all
    \def\colorrgb#1{}%
    \def\colorgray#1{}%
  \else
    % gray or color?
    \ifGPcolor
      \def\colorrgb#1{\color[rgb]{#1}}%
      \def\colorgray#1{\color[gray]{#1}}%
      \expandafter\def\csname LTw\endcsname{\color{white}}%
      \expandafter\def\csname LTb\endcsname{\color{black}}%
      \expandafter\def\csname LTa\endcsname{\color{black}}%
      \expandafter\def\csname LT0\endcsname{\color[rgb]{1,0,0}}%
      \expandafter\def\csname LT1\endcsname{\color[rgb]{0,1,0}}%
      \expandafter\def\csname LT2\endcsname{\color[rgb]{0,0,1}}%
      \expandafter\def\csname LT3\endcsname{\color[rgb]{1,0,1}}%
      \expandafter\def\csname LT4\endcsname{\color[rgb]{0,1,1}}%
      \expandafter\def\csname LT5\endcsname{\color[rgb]{1,1,0}}%
      \expandafter\def\csname LT6\endcsname{\color[rgb]{0,0,0}}%
      \expandafter\def\csname LT7\endcsname{\color[rgb]{1,0.3,0}}%
      \expandafter\def\csname LT8\endcsname{\color[rgb]{0.5,0.5,0.5}}%
    \else
      % gray
      \def\colorrgb#1{\color{black}}%
      \def\colorgray#1{\color[gray]{#1}}%
      \expandafter\def\csname LTw\endcsname{\color{white}}%
      \expandafter\def\csname LTb\endcsname{\color{black}}%
      \expandafter\def\csname LTa\endcsname{\color{black}}%
      \expandafter\def\csname LT0\endcsname{\color{black}}%
      \expandafter\def\csname LT1\endcsname{\color{black}}%
      \expandafter\def\csname LT2\endcsname{\color{black}}%
      \expandafter\def\csname LT3\endcsname{\color{black}}%
      \expandafter\def\csname LT4\endcsname{\color{black}}%
      \expandafter\def\csname LT5\endcsname{\color{black}}%
      \expandafter\def\csname LT6\endcsname{\color{black}}%
      \expandafter\def\csname LT7\endcsname{\color{black}}%
      \expandafter\def\csname LT8\endcsname{\color{black}}%
    \fi
  \fi
    \setlength{\unitlength}{0.0500bp}%
    \ifx\gptboxheight\undefined%
      \newlength{\gptboxheight}%
      \newlength{\gptboxwidth}%
      \newsavebox{\gptboxtext}%
    \fi%
    \setlength{\fboxrule}{0.5pt}%
    \setlength{\fboxsep}{1pt}%
\begin{picture}(4420.00,3400.00)%
    \gplgaddtomacro\gplbacktext{%
      \csname LTb\endcsname%%
      \put(435,380){\makebox(0,0)[r]{\strut{}\num{0}}}%
      \csname LTb\endcsname%%
      \put(435,743){\makebox(0,0)[r]{\strut{}\num{0.01}}}%
      \csname LTb\endcsname%%
      \put(435,1105){\makebox(0,0)[r]{\strut{}\num{0.02}}}%
      \csname LTb\endcsname%%
      \put(435,1468){\makebox(0,0)[r]{\strut{}\num{0.03}}}%
      \csname LTb\endcsname%%
      \put(435,1830){\makebox(0,0)[r]{\strut{}\num{0.04}}}%
      \csname LTb\endcsname%%
      \put(435,2193){\makebox(0,0)[r]{\strut{}\num{0.05}}}%
      \csname LTb\endcsname%%
      \put(435,2555){\makebox(0,0)[r]{\strut{}\num{0.06}}}%
      \csname LTb\endcsname%%
      \put(435,2918){\makebox(0,0)[r]{\strut{}\num{0.07}}}%
      \csname LTb\endcsname%%
      \put(435,3280){\makebox(0,0)[r]{\strut{}\num{0.08}}}%
      \csname LTb\endcsname%%
      \put(502,261){\makebox(0,0){\strut{}\num{0}}}%
      \csname LTb\endcsname%%
      \put(997,261){\makebox(0,0){\strut{}\num{2}}}%
      \csname LTb\endcsname%%
      \put(1493,261){\makebox(0,0){\strut{}\num{4}}}%
      \csname LTb\endcsname%%
      \put(1988,261){\makebox(0,0){\strut{}\num{6}}}%
      \csname LTb\endcsname%%
      \put(2484,261){\makebox(0,0){\strut{}\num{8}}}%
      \csname LTb\endcsname%%
      \put(2979,261){\makebox(0,0){\strut{}\num{10}}}%
      \csname LTb\endcsname%%
      \put(3475,261){\makebox(0,0){\strut{}\num{12}}}%
      \csname LTb\endcsname%%
      \put(3970,261){\makebox(0,0){\strut{}\num{14}}}%
    }%
    \gplgaddtomacro\gplfronttext{%
      \csname LTb\endcsname%%
      \put(2360,83){\makebox(0,0){\strut{}$\mathsfit t \ (\si{days})$}}%
      \csname LTb\endcsname%%
      \put(997,1830){\rotatebox{85}{\makebox(0,0){\strut{}\etiqueta{\footnotesize Mesophiles}}}}%
      \csname LTb\endcsname%%
      \put(2484,1830){\rotatebox{65}{\makebox(0,0){\strut{}\etiqueta{\footnotesize Thermophiles}}}}%
    }%
    \gplbacktext
    \put(0,0){\includegraphics{2-bacteris}}%
    \gplfronttext
  \end{picture}%
\endgroup

	\caption{Evolution of the relative mass of the populations of mesophilic and thermophilic organisms}
	\label{fig:poblacions}
\end{figure}

\begin{figure}[htb]
	\sffamily \footnotesize \centering
	%% GNUPLOT: LaTeX picture with Postscript
\begingroup
\newcommand{\etiqueta}[1]{\setlength{\fboxsep}{0.75pt}\colorbox{white}{#1}}
  \makeatletter
  \providecommand\color[2][]{%
    \GenericError{(gnuplot) \space\space\space\@spaces}{%
      Package color not loaded in conjunction with
      terminal option `colourtext'%
    }{See the gnuplot documentation for explanation.%
    }{Either use 'blacktext' in gnuplot or load the package
      color.sty in LaTeX.}%
    \renewcommand\color[2][]{}%
  }%
  \providecommand\includegraphics[2][]{%
    \GenericError{(gnuplot) \space\space\space\@spaces}{%
      Package graphicx or graphics not loaded%
    }{See the gnuplot documentation for explanation.%
    }{The gnuplot epslatex terminal needs graphicx.sty or graphics.sty.}%
    \renewcommand\includegraphics[2][]{}%
  }%
  \providecommand\rotatebox[2]{#2}%
  \@ifundefined{ifGPcolor}{%
    \newif\ifGPcolor
    \GPcolortrue
  }{}%
  \@ifundefined{ifGPblacktext}{%
    \newif\ifGPblacktext
    \GPblacktextfalse
  }{}%
  % define a \g@addto@macro without @ in the name:
  \let\gplgaddtomacro\g@addto@macro
  % define empty templates for all commands taking text:
  \gdef\gplbacktext{}%
  \gdef\gplfronttext{}%
  \makeatother
  \ifGPblacktext
    % no textcolor at all
    \def\colorrgb#1{}%
    \def\colorgray#1{}%
  \else
    % gray or color?
    \ifGPcolor
      \def\colorrgb#1{\color[rgb]{#1}}%
      \def\colorgray#1{\color[gray]{#1}}%
      \expandafter\def\csname LTw\endcsname{\color{white}}%
      \expandafter\def\csname LTb\endcsname{\color{black}}%
      \expandafter\def\csname LTa\endcsname{\color{black}}%
      \expandafter\def\csname LT0\endcsname{\color[rgb]{1,0,0}}%
      \expandafter\def\csname LT1\endcsname{\color[rgb]{0,1,0}}%
      \expandafter\def\csname LT2\endcsname{\color[rgb]{0,0,1}}%
      \expandafter\def\csname LT3\endcsname{\color[rgb]{1,0,1}}%
      \expandafter\def\csname LT4\endcsname{\color[rgb]{0,1,1}}%
      \expandafter\def\csname LT5\endcsname{\color[rgb]{1,1,0}}%
      \expandafter\def\csname LT6\endcsname{\color[rgb]{0,0,0}}%
      \expandafter\def\csname LT7\endcsname{\color[rgb]{1,0.3,0}}%
      \expandafter\def\csname LT8\endcsname{\color[rgb]{0.5,0.5,0.5}}%
    \else
      % gray
      \def\colorrgb#1{\color{black}}%
      \def\colorgray#1{\color[gray]{#1}}%
      \expandafter\def\csname LTw\endcsname{\color{white}}%
      \expandafter\def\csname LTb\endcsname{\color{black}}%
      \expandafter\def\csname LTa\endcsname{\color{black}}%
      \expandafter\def\csname LT0\endcsname{\color{black}}%
      \expandafter\def\csname LT1\endcsname{\color{black}}%
      \expandafter\def\csname LT2\endcsname{\color{black}}%
      \expandafter\def\csname LT3\endcsname{\color{black}}%
      \expandafter\def\csname LT4\endcsname{\color{black}}%
      \expandafter\def\csname LT5\endcsname{\color{black}}%
      \expandafter\def\csname LT6\endcsname{\color{black}}%
      \expandafter\def\csname LT7\endcsname{\color{black}}%
      \expandafter\def\csname LT8\endcsname{\color{black}}%
    \fi
  \fi
    \setlength{\unitlength}{0.0500bp}%
    \ifx\gptboxheight\undefined%
      \newlength{\gptboxheight}%
      \newlength{\gptboxwidth}%
      \newsavebox{\gptboxtext}%
    \fi%
    \setlength{\fboxrule}{0.5pt}%
    \setlength{\fboxsep}{1pt}%
\begin{picture}(4420.00,3400.00)%
    \gplgaddtomacro\gplbacktext{%
      \csname LTb\endcsname%%
      \put(487,380){\makebox(0,0)[r]{\strut{}\num{0}}}%
      \csname LTb\endcsname%%
      \put(487,1070){\makebox(0,0)[r]{\strut{}\num{0.5}}}%
      \csname LTb\endcsname%%
      \put(487,1761){\makebox(0,0)[r]{\strut{}\num{1}}}%
      \csname LTb\endcsname%%
      \put(487,2451){\makebox(0,0)[r]{\strut{}\num{1.5}}}%
      \csname LTb\endcsname%%
      \put(487,3142){\makebox(0,0)[r]{\strut{}\num{2}}}%
      \csname LTb\endcsname%%
      \put(554,261){\makebox(0,0){\strut{}\num{0}}}%
      \csname LTb\endcsname%%
      \put(1043,261){\makebox(0,0){\strut{}\num{2}}}%
      \csname LTb\endcsname%%
      \put(1531,261){\makebox(0,0){\strut{}\num{4}}}%
      \csname LTb\endcsname%%
      \put(2020,261){\makebox(0,0){\strut{}\num{6}}}%
      \csname LTb\endcsname%%
      \put(2508,261){\makebox(0,0){\strut{}\num{8}}}%
      \csname LTb\endcsname%%
      \put(2997,261){\makebox(0,0){\strut{}\num{10}}}%
      \csname LTb\endcsname%%
      \put(3485,261){\makebox(0,0){\strut{}\num{12}}}%
      \csname LTb\endcsname%%
      \put(3974,261){\makebox(0,0){\strut{}\num{14}}}%
    }%
    \gplgaddtomacro\gplfronttext{%
      \csname LTb\endcsname%%
      \put(100,1830){\rotatebox{-270}{\makebox(0,0){\strut{}$\mathsfit P \ (\si{MW})$}}}%
      \csname LTb\endcsname%%
      \put(2386,83){\makebox(0,0){\strut{}$\mathsfit t \ (\si{days})$}}%
      \colorrgb{0.58,0.00,0.83}%%
      \put(920,1485){\rotatebox{-87}{\makebox(0,0){\strut{}\etiqueta{\footnotesize Mesophiles}}}}%
      \colorrgb{1.00,0.65,0.00}%%
      \put(1775,615){\rotatebox{12}{\makebox(0,0){\strut{}\etiqueta{\footnotesize Thermophiles}}}}%
    }%
    \gplbacktext
    \put(0,0){\includegraphics{2-potencies}}%
    \gplfronttext
  \end{picture}%
\endgroup

	\caption{Power output of the mesophilic and thermophilic reactions}
	\label{fig:potencies}
\end{figure}

\clearpage
\appendix
\section{References and Further Reading}
\begin{btSect}{fonts}
	\bibliographystyle{ieeetr}
	\subsection*{References}	
	\btPrintCited
\end{btSect}

\begin{btSect}{UphysicscNotes}
	\bibliographystyle{ieeetr}
	\subsection*{Further Reading}	
	\btPrintNotCited
\end{btSect}

\end{document}
